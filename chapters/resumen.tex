\chapter*{Resumen}
\chaptermark{Resumen}
\addcontentsline{toc}{chapter}{Resumen} 

La Esclerosis Múltiple (EM) es la enfermedad neurológica crónica incapacitante más común del sistema nervioso central, en donde el recubrimiento aislante de las células nerviosas en la médula espinal y el cerebro están dañadas. La EM se caracteriza por la presencia de lesiones en el cerebro, predominantemente en el tejido de la sustancia blanca. Gracias a la sensibilidad de la resonancia magnética (RM) para mostrar la actividad focal de las lesiones y el progreso de la enfermedad, la RM se ha convertido en una herramienta esencial para el diagnóstico y la evaluación de la EM. Igualmente, se ha demostrado que la atrofia del tejido cerebral medida a través de la RM está relacionada con el incremento de la discapacidad, mostrando que la pérdida de tejido es un componente importante de la progresión de la enfermedad.

La correlación existente entre la atrofia del tejido cerebral y el estado de incapacidad de la enfermedad, ha aumentado la necesidad de desarrollar 
herramientas automáticas de segmentación capaces de medir de forma precisa el volumen de los tejidos cerebrales. Sin embargo, la segmentación automática del tejido cerebral sigue siendo un problema complicado, fundamentalmente debido a factores como la complejidad de las imágenes, las diferencias en las intensidades de tejido, el ruido de las imágenes, las diferencias en la homogeneidad de las adquisiciones o la ausencia de modelos anatómicos capaces de modelar cada una de las estructuras del cerebro. Asimismo, se ha demostrado también que las lesiones de sustancia blanca reducen la precisión de los métodos automáticos de segmentación, subrayando así la necesidad de procesar las lesiones antes de la segmentación utilizando un proceso conocido como \textit{lesion filling}. No obstante, el proceso de \textit{lesion filling} requiere que las máscaras de lesión sean conocidas a priori, lo que puede ser difícil de conseguir, conllevando tiempo y siendo propenso a variabilidad entre radiólogos. Este hecho y la necesidad de analizar las lesiones de EM tanto en estudios individuales como temporales ha llevado al desarrollo de un 
gran número de métodos automáticos de segmentación de las lesiones. 

El objetivo principal de esta tesis es el desarrollo de un nuevo método de segmentación totalmente automático capaz de medir con precisión el volumen cerebral en imágenes de pacientes de EM con lesiones. Para conseguirlo, en esta tesis nos hemos concentrado en cada uno de los procesos encadenados necesarios para desarrollar tal método. Primero, hemos realizado un resumen cualitativo y cuantitativo de las técnicas de segmentación ya existentes utilizando diferentes conjuntos de imágenes de sujetos sanos, con el objetivo de entender las ventajas e inconvenientes de cada técnica. Los resultados obtenidos demuestran que los métodos que incorporan información a priori de tipo morfológica o de contexto local tienden a ser menos proclives a los cambios en la adquisición de las secuencias o  en las homogeneidades de las intensidades, en comparación con métodos más simples basados solamente en intensidad.

En segundo lugar, hemos estudiado y analizado el efecto que producen las lesiones de sustancia blanca en la segmentación de imágenes de pacientes de EM. Para ello, hemos realizado varios experimentos utilizando bases de datos de 1.5T adquiridas en diferentes escáneres con el fin de analizar el efecto de la intensidad y el volumen de las lesiones en las diferencias en volumen cerebral de varios métodos de segmentación de tejido. En todos los métodos, la inclusión de las lesiones en el proceso de segmentación no sólo introduce errores en las mediciones del volumen total de tejido debido a los vóxeles de las lesiones que fueron mal clasificados, sino que también tienen un efecto claro en las diferencias de volumen observadas en el tejido sano. Este efecto es menos relevante en los métodos que incorporan información a priori de tipo morfológica o de contexto local.

En tercer lugar, nos hemos concentrado en el proceso de \textit{lesion filling}, donde hemos resumido y analizado la precisión de las diferentes técnicas propuestas en el campo. Este análisis nos ha servido de base para proponer una nueva técnica de \textit{lesion filling} que mejore las limitaciones observadas en los métodos anteriores. Los resultados obtenidos muestran que en comparación con el resto de métodos propuestos, nuestro método es efectivo con diferentes tipos de imágenes e independientemente del método de segmentación utilizado a continuación. 

Seguidamente, hemos realizado un análisis completo de los efectos de automatizar la segmentación de las lesiones de sustancia blanca y el \textit{lesion filling}  
en la posterior segmentación del tejido cerebral. Para ello, hemos evaluado la eficacia de dos sistemas automáticos que incorporan estos procesos con el fin de entender el papel de las lesiones residuales que no fueron detectadas y, consecuentemente no procesadas, en las diferencias de volumen cerebral observadas. Nuestros resultados muestran que los sistemas donde la segmentación de las lesiones y el \textit{lesion filling} fue automático reducen significativamente el impacto de las lesiones de sustancia blanca en la segmentación del tejido, mostrando un eficacia similar a los sistemas con intervención manual de los expertos. 

Cada una de estas fases nos ha servido de base para el desarrollo de un nuevo método de segmentación multicanal diseñado con el objetivo de segmentar imágenes de RM de pacientes de EM.  El método que hemos propuesto se ha desarrollado e implementado integrando no sólo la información proveniente de la intensidad de los vóxeles, sino a través de la incorporación de atlas morfológicos y estructurales que guían la segmentación del tejido. Los vóxeles candidatos de ser lesiones son estimados y procesados antes de la segmentación del tejido utilizando un algoritmo de postproceso basado en la información del contexto local y la información anatómica y morfológica previa. Este método de segmentación ha sido evaluado de forma cuantitativa y cualitativa usando diferentes conjuntos de imágenes que contenían lesiones de sustancia blanca. Los resultados muestran que la precisión del método propuesto es consistente y muy competitiva en todo tipo de imágenes en comparación con otras técnicas propuestas. En este sentido, los porcentajes de error obtenidos en los diferentes experimentos llevados a cabo muestran que el método propuesto mejora la segmentación del tejido cerebral de las imágenes con lesiones.

Esta tesis doctoral forma parte de varios proyectos que nuestro grupo de investigación está llevando a cabo en colaboración con los diferentes centros hospitalarios involucrados. Como parte de estos objetivos, todos los programas desarrollados durante esta tesis se han hecho públicos para el libre uso de la comunidad científica. En el caso del método de \textit{lesion filling}, éste ya está siendo utilizado en los hospitales colaboradores. Pensamos igualmente que el método de segmentación propuesto será también útil en futuros entornos de investigación y ensayos clínicos.

 
%%% Local Variables:
%%% mode: latex
%%% TeX-master: "../main"
%%% End:
