%Conclusions

\chapter{Conclusions}
% also add contributions

This thesis synthesizes our work done during the last three years. Following the same objectives defined in the Introduction chapter, we summarize in what follows the main conclusions and contributions of this thesis: 

\begin{itemize}

\item We analyzed and evaluated the state-of-the-art of brain tissue segmentation methods. This first stage aimed to quantitative review and evaluate different proposed tissue segmentation techniques in order to understand their advantages and drawbacks. As part of the resulting analysis published in the\textit{ Journal of Magnetic Resonance Imaging} in January of 2015,  
\textbf{our results showed a higher accuracy on several methods that incorporated morphological prior information and/or spatial constraints such as FAST, SPVASEG and SPM8. These methods were also less prone to changes in acquisition sequences and intensity inhomogeneities}.

\item We studied the effect of WM lesions on tissue segmentation of MS patient images. The second stage to cover was focused on the analysis of the effects of WM lesions on the tissue distributions. Six of the analyzed methods on Chapter \ref{chapter:chapter_2} were evaluated on multi-center 1.5T MS data from different scanners. Related to the previous stage, our results stressed \textbf{the necessity of adding morphological prior information and/or spatial constraints in automated brain tissue segmentation, not only to overcome inherent MRI artifacts but also as an important component to deal with WM lesions}. Furthermore, our analysis of the effects of WM lesions on tissue volume showed that \textbf{the inclusion of WM lesions on tissue segmentation not only biased the total tissue volume measurements by the addition of miss-classified lesion voxels, but also by the effect of these lesions in observed differences in normal-appearing tissue volume.} The entire analysis was published in the \textit{American Journal of Neuroradiology} in February 2015.

\item We proposed a new technique to reduce the effects of WM lesions on tissue segmentation of MS patient images. The third stage required first to compare the accuracy of different proposed lesion filling techniques in the literature with the aim to propose then a new technique to reduce the effects of WM lesions on tissue segmentation. \textbf{The lesion filling method proposed in this thesis, was shown effective in different data and independently of the tissue segmentation method used afterwards. The proposed approach outperformed the rest of methods on both 1.5T and 3T data when FAST was used, while its performance was similar or lower to the best available strategy when SPM8 was used.} The proposed lesion filling method was  published in the\textit{ NeuroImage: Clinical} journal in August of 2014. \textbf{Furthermore, we released a public version on the proposed method that can be freely downloaded from our research team web page}\footnote{The latest version on the proposed lesion filling method can be download from \texttt{http://atc.udg.edu/nic/slfToolbox/index.html}}. This software is already been used in the collaborating hospitals.  

\item We analyzed and evaluated the effect of automated WM lesion segmentation and filling on the tissue segmentation. During the fourth stage proposed, we quantitatively evaluated the accuracy of two state-of-the-art automated pipelines that incorporate unsupervised lesion segmentation, lesion filling and tissue segmentation on MS data.  As shown in the published paper in the \textit{NeuroImage: Clinical journal} in October of 2015, our analysis showed that \textbf{pipelines that incorporated automated lesion segmentation and filling were capable to reduce significantly the impact of WM lesions on tissue segmentation, performing similarly to the pipelines that required manual expert intervention}.

\item Finally, we proposed a new fully automated tissue segmentation method for MS patient images containing lesions. The main goal of the thesis was to propose a fully automated tissue segmentation method capable to deal with images of MS patients. As shown in Chapter \ref{chapter:chapter_6}, the proposed method incorporated all the major insights obtained from previous stages with the aim of provide a robust fully automated tissue approach for accurate brain volume measurements. Our results showed that  \textbf{when compared with existent tissue segmentation methods, the presented approach yielded a higher accuracy in tissue segmentation while the influence of MS lesions on tissue segmentation was lower or similar to the best state-of-the-art pipeline incorporating automated lesion segmentation and filling}. This work has been submitted for publication in the\textit{ Medical Image Analysis journal }in January 2016. \textbf{As part of this work, we also released a public version on the proposed method that can be freely downloaded from our research team web page}\footnote{A public version of the method can be download from \texttt{http://atc.udg.edu/nic/msseg/index.html}}.

\end{itemize}

During this PhD thesis, different collaborations have been done with other researchers of the VICOROB group. First, we evaluated the effect of MS lesions on longitudinal registration in the published study of Diez et al. \cite{Diez2014}, where we contributed with several processing steps such as lesion filling. Lately, we were also involved in the development of several automated lesion segmentation pipelines that allowed us to gain knowledge about this topic. In this regard, we helped to implement two different lesion segmentation pipelines for MS such as those published in the papers of Cabezas et al. \cite{Cabezas2014b} and Roura et al. \cite{Roura2015}, respectively. Furthermore, we also collaborated on a new pipeline for automated lesion segmentation of Lupus lesions proposed by Roura et al., which has been submitted for publication recently. 


\section{Future work}

Unfortunately, there are several aspects that have been not investigated during this thesis. One of the main limitations on several stages has been the lack of 3T images with high lesion load. As pointed out in Chapters \ref{chapter:chapter_5} and \ref{chapter:chapter_6}, the low mean lesion load of the cohorts analyzed, which indeed has the major interest for the medical experts, has not allowed to investigate better the performance of the analyzed pipelines in the presence of images with higher lesion load. In the case of our tissue segmentation method, we believe that an additional analysis of the performance with images with higher lesion load would be helpful not only to analyze the robustness of the proposed algorithm, but also to investigate the benefits of adding other image sequences such as T2-w or PD-w. 

Although the proposed tissue segmentation method has been designed for cross-sectional data, there is an increasingly clinical interest in the measurements of longitudinal changes in tissue volume. We believe that the proposed method could be extended to longitudinal changes by re-adapting the pipeline with prior registering of time point images before tissue segmentation. This is in fact one of the goals that our team has in mind to tackle first within the research framework of the BiomarkEM.cat project, in order to release suitable tools that can be used in clinical settings. 

The ultimate goal should be to provide state-of-the-art tools for the collaborating hospitals involved in these research projects  that may be useful not only to diagnose and  monitorize the progression of disease, but also to evaluate new treatments in MS patients.  Related to that, the tools developed in this thesis should be integrated with other tools developed in our group in order to implement this complete system capable to provide robust useful biomarkers in MS such as the number of lesions, lesion volume, brain tissue volume or brain atrophy. 

%%% Local Variables:
%%% mode: latex
%%% TeX-master: "../main"
%%% End:
