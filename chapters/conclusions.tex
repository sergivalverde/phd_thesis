\chapter{Conclusions}

This thesis synthesizes our work done over the last three years. Following the same objectives defined in the Introduction, in what follows  we summarize the main conclusions and contributions of this thesis: 

\begin{itemize}

\item We analyzed and evaluated the state of the art in brain tissue segmentation methods. This first sub-objective aimed to quantitatively review and evaluate different proposed tissue segmentation techniques in order to understand their advantages and drawbacks. As part of the resulting analysis published in the\textit{ Journal of Magnetic Resonance Imaging} in January of 2015,   \textbf{our results showed a higher accuracy with several methods that incorporated morphological prior information and/or spatial constraints such as the FAST, SPVASEG or SPM8. These methods were also less prone to changes in acquisition sequences and intensity inhomogeneities}.

\item We studied the effect of WM lesions on tissue segmentation of MS patient images. The second sub-objective to cover focused on the analysis of the effects of WM lesions on the tissue distributions. Six of the analyzed methods in Chapter \ref{chapter:chapter_2} were evaluated with multi-center 1.5T MS data from different scanners. Related to the previous sub-objective, our results stressed \textbf{the necessity of adding morphological prior information and/or spatial constraints to the automated brain tissue segmentation, not only to overcome inherent MRI artifacts but also as an important component of dealing with WM lesions}. Furthermore, our analysis of the effects of WM lesions on tissue volume showed that \textbf{the inclusion of WM lesions in tissue segmentation not only biased the total tissue volume measurements by the addition of misclassified lesion voxels, but also by the effect of these lesions on differences observed in normal-appearing tissue volume.} The entire analysis was published in the \textit{American Journal of Neuroradiology} in February 2015.

\item We proposed a new technique to reduce the effects of WM lesions on tissue segmentation of MS patient images. The third sub-objective first required comparing the accuracy of different  lesion filling techniques proposed in the literature with the aim of then proposing a new technique to reduce the effects of WM lesions on tissue segmentation. \textbf{The lesion filling method proposed in this thesis was shown to be effective with different data and independent of the tissue segmentation method used afterwards. Our approach outperformed the rest of methods with both 1.5T and 3T data when the FAST was used, while its performance was similar to or lower than the best available strategy when the SPM8 was used.} The proposed lesion filling method was  published in the\textit{ NeuroImage: Clinical} journal in August of 2014. \textbf{Furthermore, we released a public version of the proposed method that can be downloaded for free from our research team web page}\footnote{The latest version of the proposed lesion filling method can be downloaded from \texttt{http://atc.udg.edu/nic/slfToolbox}}. This software is already being used in the collaborating hospitals.  

\item We analyzed and evaluated the effect of automated WM lesion segmentation and filling on the tissue segmentation. In the fourth sub-objective, we quantitatively evaluated the accuracy of two state-of-the-art automated pipelines that incorporate unsupervised lesion segmentation, lesion filling and tissue segmentation with MS data.  As shown in the paper published in the \textit{NeuroImage: Clinical} journal in October of 2015, our analysis showed that \textbf{pipelines that incorporated automated lesion segmentation and filling were capable of significantly reducing the impact of WM lesions on tissue segmentation, performing similarly to pipelines that required expert manual intervention}.

\item Finally, we proposed a new fully automated tissue segmentation method for MS patient images containing lesions. The main goal of this thesis was to propose a fully automated tissue segmentation method capable of dealing with images from MS patients. As shown in Chapter \ref{chapter:chapter_6}, the proposed method incorporates all the major insights obtained from previous sub-objectives with the aim of providing a robust, fully automated tissue approach for accurate brain volume measurements. Our results showed that  \textbf{when compared with existing tissue segmentation methods, the presented approach yielded a higher accuracy in tissue segmentation while the influence of MS lesions on tissue segmentation was lower or similar to the best state-of-the-art pipeline incorporating automated lesion segmentation and filling}. This work has been submitted for publication in the\textit{ Medical Image Analysis} journal in January 2016. \textbf{As part of this work, we also released a public version of the proposed method that can be downloaded for free from our research team web page}\footnote{A public version of the method can be downloaded from \texttt{http://atc.udg.edu/nic/msseg}}.

\end{itemize}

Throughout this PhD thesis, various collaborations have taken place with other researchers of the VICOROB group. First, we evaluated the effect of MS lesions on longitudinal registration in the published study of Diez et al. \cite{Diez2014}, where we contributed several processing steps, including lesion filling. More recently, we were also involved in the development of several automated lesion segmentation pipelines that allowed us to gain knowledge on this topic. In this regard, we helped to implement two different lesion segmentation pipelines for MS, which were published in the papers of Cabezas et al. \cite{Cabezas2014b} and Roura et al. \cite{Roura2015}, respectively. Furthermore, we also collaborated on a new pipeline for automated lesion segmentation of lupus lesions proposed by Roura et al., which was submitted for publication recently. 

\section{Future work}

Unfortunately, there are several aspects that have not been investigated during this thesis. One of the main limitations in several sub-objectives has been the lack of 3T images with high lesion loads. As pointed out in Chapters \ref{chapter:chapter_5} and \ref{chapter:chapter_6}, the low mean lesion load of the cohorts analyzed, which indeed has been the major interest for medical experts, has not allowed us to better investigate the performance of the analyzed pipelines in the presence of images with higher lesion loads. In the case of our tissue segmentation method, we believe that an additional analysis of the performance with images with higher lesion loads would be helpful not only to analyze the robustness of the proposed algorithm, but also to investigate the benefits of adding other image sequences such as T2-w or PD-w. 

Although the proposed tissue segmentation method has been designed for cross-sectional data, there is an increasing clinical interest in the measuring of longitudinal changes in tissue volume. We believe that the proposed method could be extended to longitudinal changes by re-adapting the pipeline with prior registering of time point images before the tissue segmentation. This is in fact one of the goals that our team has in mind to tackle first within the research framework of the BiomarkEM.cat project, in order to release suitable tools that can be used in clinical settings. 

The ultimate goal should be to provide state-of-the-art tools for the collaborating hospitals involved in these research projects  that may be useful not only to diagnose and  monitor the progression of this disease, but also to evaluate new treatments for MS patients.  Related to that, the tools developed in this thesis should be integrated with other tools developed in our group in order to implement this complete system capable of providing robust and useful biomarkers in MS such as the number of lesions, lesion volume, brain tissue volume or brain atrophy. 

%%% Local Variables:
%%% mode: latex
%%% TeX-master: "../main"
%%% End: