\chapter*{Resum}

\addcontentsline{toc}{chapter}{Resum} 

L'Esclerosi Múltiple (EM)  és la malaltia neurològica crònica incapacitant més comuna del sistema nerviós central, on el recobriment aïllant de les cèl·lules nervioses a la medul·la espinal i el cervell estan danyades. L'EM es caracteritza per la presència de lesions en el cervell, predominantment en el jteixit de la substància blanca. Gràcies a la seva sensibilitat per mostrar l'activitat focal de les lesions i el progrés de la malaltia, la ressonància magnètica (RM) s'ha convertit en una eina essencial per al diagnòstic i l'avaluació de l'EM. Igualment, s'ha demostrat que l'atròfia del teixit cerebral mesurada a través de la RM està relacionada amb l'increment de la discapacitat, mostrant que la pèrdua de teixit és un component important de la progressió de la malaltia.

La correlació existent entre l'atròfia del teixit cerebral i l'estat d'incapacitat de la malaltia, ha augmentat la necessitat de desenvolupar
eines automàtiques de segmentació amb capacitat per mesurar de forma precisa el volum dels teixits cerebrals. No obstant, la segmentació automàtica del teixit cerebral segueix sent un problema complicat, fonamentalment a causa de factors com la complexitat de les imatges, les diferències en les intensitats de teixit, el soroll de les imatges, les diferències en l'homogeneïtat de les adquisicions o l'absència de models anatòmics capaços de modelar cadascuna de les estructures del cervell. Així mateix, s'ha demostrat també que les lesions de substància blanca redueixen la precisió dels mètodes automàtics de segmentació, subratllant així la necessitat de processar les lesions abans de la segmentació utilitzant un procés conegut com \textit{lesion filling}. Tanmateix, el procés de \textit{lesion filling} requereix que les màscares de lesió siguin conegudes a priori, el que pot ser difícil d'aconseguir, comportant temps i sent propens a variabilitat entre radiòlegs. Aquest fet i la necessitat d'analitzar les lesions d'EM tant en estudis individuals com temporals ha portat al desenvolupament d'un gran nombre de mètodes automàtics de segmentació de les lesions.

L'objectiu principal d'aquesta tesi és el desenvolupament d'un nou mètode de segmentació totalment automàtic capaç de mesurar amb precisió el volum cerebral en imatges de pacients d'EM amb lesions. Per aconseguir-ho, en aquesta tesi ens hem concentrat en cadascun dels processos necessaris per a desenvolupar aquest mètode. Primer, hem fet un resum qualitatiu i quantitatiu de les tècniques de segmentació ja existents utilitzant diferents conjunts d'imatges de subjectes sans, amb l'objectiu d'entendre els avantatges i inconvenients de les diferents tècniques. Els resultats obtinguts demostren que els mètodes que incorporen informació a priori de tipus morfològica o de context local tendeixen a ser menys proclius als canvis en l'adquisició de les seqüències o en les homogeneïtats de les intensitats, en comparació amb mètodes més simples basats només en intensitat.

En segon lloc, hem estudiat i analitzat l'efecte que produeixen les lesions de substància blanca a la segmentació d'imatges de pacients d'EM. A tal fi, hem realitzat diversos experiments utilitzant bases de dades de 1.5T adquirides en diferents escàners per tal d'analitzar l'efecte de la intensitat i el volum de les lesions en les diferències en volum cerebral de diversos mètodes de segmentació de teixit. En tots els mètodes, la inclusió de les lesions en el procés de segmentació no només  introdueix errors en els mesuraments del volum total de teixit a causa dels vòxels de les lesions mal classificats, sinó que també té un efecte clar en les diferències de volum en el teixit sa. Aquest efecte és menys rellevant en els mètodes que incorporen informació a priori de tipus morfològica o de context local.

En tercer lloc, ens hem concentrat en el procés de \textit{lesion filling}, on hem resumit i analitzat la precisió de les diferents tècniques proposades en el camp. Aquesta anàlisi ens ha servit de base per proposar una nova tècnica de \textit{lesion filling} que millori les limitacions observades en els mètodes anteriors. Els resultats obtinguts mostren que, en comparació amb la resta de mètodes proposats, el nostre mètode és efectiu amb diferents tipus d'imatges i independentment del mètode de segmentació utilitzat a continuació.

Seguidament, hem realitzat una anàlisi completa dels efectes d'automatitzar la segmentació de les lesions de substància blanca i el \textit{lesion filling}
en la posterior segmentació del teixit cerebral. Per això, hem avaluat l'eficàcia de dos sistemes automàtics que incorporen aquests processos per tal d'entendre el paper de les lesions residuals que no van ser detectades i, per tant no processades, en les diferències de volum cerebral. Els nostres resultats mostren que els sistemes on la segmentació de les lesions i el \textit{lesion filling} va ser automàtic redueixen significativament l'impacte de les lesions de substància blanca a la segmentació del teixit, mostrant un eficàcia similar als sistemes amb intervenció manual dels experts.

Cadascuna d'aquestes fases ens ha servit de base per al desenvolupament d'un nou mètode de segmentació multi-canal dissenyat amb l'objectiu de segmentar imatges de RM de pacients d'EM. El mètode que hem proposat s'ha desenvolupat i implementat integrant no només la informació provinent de la intensitat dels vòxels, sinó a través de la incorporació d'atles morfològics i estructurals que guien la segmentació del teixit. Els vòxels candidats de ser lesions són estimats i processats abans de la segmentació del teixit utilitzant un algoritme de post-processat basat en la informació del context local i la informació anatòmica i morfològica prèvia. Aquest mètode de segmentació ha estat avaluat de forma quantitativa i qualitativa utilitzant diferents conjunts d'imatges que contenen lesions de substància blanca. Els resultats mostren que la precisió del mètode proposat és consistent i molt competitiva en tot tipus d'imatges en comparació amb altres tècniques proposades. En aquest sentit, els percentatges d'error obtinguts en els diferents experiments duts a terme mostren que el mètode proposat millora la segmentació del teixit cerebral de les imatges amb lesions.

Aquesta tesi doctoral forma part de diversos projectes que el nostre grup de recerca està duent a terme en col·laboració amb els diferents centres hospitalaris involucrats. Com a part d'aquests objectius, tots els programes desenvolupats durant aquesta tesi s'han fet públics per al lliure ús de la comunitat científica. En el cas del mètode de \textit{lesion filling}, aquest ja està sent utilitzat en els hospitals col·laboradors. Pensem igualment que el mètode de segmentació proposat serà també útil en futurs entorns d'investigació i assajos clínics.
%%% Local Variables:
%%% mode: latex
%%% TeX-master: "../main"
%%% End:
