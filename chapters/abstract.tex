\chapter*{Abstract}

\addcontentsline{toc}{chapter}{Abstract} 

Nowadays, Multiple Sclerosis (MS) is the most frequent non-traumatic neurological disease that causes more disability in young adults. MS is the most common chronic immune-mediated disabling neurological disease of the Central Nervous System, in which the insulating covers of the nerve cells in the spinal chord and brain are damaged. MS is characterized by the presence of lesions in the brain, predominantly in the white matter (WM) tissue of the brain. Due to the sensitivity of Magnetic Resonance Imaging (MRI) to reveal focal WM lesions and disease activity in time and space, MRI has become an essential tool for the diagnosis and evaluation of the MS disease. Furthermore, MRI brain tissue atrophy measures have been shown to correlate with the disability status, showing that tissue loss is an important component of the disease progression. 

The existent correlation between brain tissue atrophy measures and MS disability status has increased the necessity to develop robust automated brain tissue segmentation methods capable to perform accurate brain tissue volume measurements. However, automated segmentation of brain tissue is still a challenging problem due to the complexity of the images, differences in tissue intensities, noise, intensity inhomogeneities and the absence of models of the anatomy that fully capture the possible deformations in each structure. Moreover, it has been shown that WM lesions reduce the accuracy of automated tissue segmentation methods, which highlights the necessity to process these lesions before tissue segmentation, a process which is known as lesion filling. However, lesion filling requires to manually annotate lesions before tissue segmentation, which is time-consuming, prone to variability between expert radiologists, or not always available. This fact and the need to analyze quantitatively focal MS lesions in individual and temporal studies has led to the development of a wide number of automated lesion segmentation of MS lesions. 

% main goal and different things done to accomplish this.
The main goal of this thesis is to develop a novel fully automated brain tissue segmentation method capable of computing accurate tissue volume measurements on images of MS patients containing lesions. In order to fulfill this goal, in this thesis we have focused on each of the concatenated processes that are necessary to develop a fully automated tissue segmentation method. Firstly, we have analyzed and evaluated the state-of-the-art of tissue segmentation methods on data from healthy subjects, where we have performed a quantitative review of the different proposed tissue segmentation techniques, with the aim to understand their advantages and drawbacks. Our experimental results have shown that methods that incorporate morphological prior information and/or spatial constraints are less prone to changes in acquisition sequences and intensity inhomogeneities, when compared with simpler strategies intensity based methods.   

In a second stage, we have studied and evaluated the effect of WM lesions on tissue segmentation of MS patient images. In this regard, we have performed several experiments using multi-center 1.5T MS data from different scanners in order to analyze the effects of lesion signal intensity and lesion size on the differences in tissue volume of several tissue segmentation methods. In all methods, obtained results have indicated that the inclusion of WM lesions on tissue segmentation not only biased the total tissue volume measurements by the addition of miss-classified lesion voxels, but also had a direct effect on the observed differences in normal-appearing tissue. This effect has been less relevant in those methods that incorporate prior information and/or spatial context. 

In a third stage, we have focused on lesion filling, reviewing and analyzing the accuracy of different proposed lesion filling techniques in the literature. From this results, we have proposed a new lesion filling technique with the aim to overcome the limitations of previously proposed methods. When compared with these methods, our experimental results have shown that the proposed lesion filling method is effective  with different databases and independently of the tissue segmentation method used afterwards. 

Afterwards, we have focused on a comprehensive analysis of the effects of automated lesion segmentation and filling in tissue segmentation. We have evaluated the accuracy of two pipelines that incorporated automated lesion segmentation, lesion filling and tissue segmentation on MS data, with the aim to understand the extend of the effect of remaining WM lesions on the differences in tissue segmentation. Our findings have evidenced that up to certain lesion load, pipelines that incorporated automated lesion segmentation and filling are capable to reduce significantly the impact of WM lesions on tissue segmentation, showing a similar performance to the pipelines where expert lesion annotations were used.

All these stages has served as a basis to develop a novel multi-channel method designed to segment brain tissues in MRI images of MS patients. The proposed tissue segmentation method has been designed and implemented using a combination of intensity, anatomical and morphological prior maps to guide the tissue segmentation. WM outliers have been estimated and filled before segmentation using a multi-channel post-processing rule-based algorithm using spatial context, and prior anatomical and morphological atlases. The proposed method has been quantitatively and qualitatively evaluated using different databases of images containing WM lesions, yielding competitive and consistent results in both general and MS specific databases. The percentages of error obtained in the different experiments carried out show that the proposed algorithm effectively improves automated brain tissue segmentation in images containing lesions. 

This PhD thesis is part of several existing project frameworks carried out by our research group in collaboration with different hospital centers. As part of the goals of these research projects, software implementations of all the proposed methods in this thesis have been released for public use of the research community. The proposed lesion filling method is currently being used by the collaborating hospitals. We believe that the proposed fully automated tissue segmentation method will be also beneficial in clinical settings.  






%%% Local Variables:
%%% mode: latex
%%% TeX-master: "../main"
%%% End:




