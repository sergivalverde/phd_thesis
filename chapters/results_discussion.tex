\chapter{Main results and discussion}
%\addcontentsline{toc}{chapter}{Main results and discussion} 

%Un capítol en el qual se sintetitzin els principals resultats i la discussió %d’aquests resultats.

MRI tissue segmentation techniques are increasingly being used as the standard tools to assess brain tissue volume. However, automated tissue segmentation is still a challenging task in MS, due to tissue abnormalities found in MS image patients such as WM lesions that are known to reduce the accuracy of tissue segmentation methods. When expert manual annotations of WM lesions are available, lesion filling has been shown to be an effective method to reduce the effects of those lesions on tissue segmentation. However, manual annotations are time-consuming and prone to variability between experts, which in combination of the need to analyze quantitatively focal MS lesions in individual and temporal studies has led to the development of a wide number of automated lesion segmentation of MS lesions. Therefore, a solid understanding of the effects of MS lesions on automated pipelines that concatenate processes such as lesion segmentation, lesion filling and tissue segmentation is important. 

As said in section \ref{sec:objectives}, each of these processes cover part of the necessary knowledge needed to tackle the problem of automated brain tissue segmentation of images containing lesions. This chapter provides a comprehensive discussion of the main results obtained on previous chapters, analyzing each of these necessary processes until the development of a fully automated tissue segmentation method for MS images.

\section{Effect of WM lesions on tissue segmentation}

A wide number of automated tissue segmentation methods have been proposed in the literature so far. In Chapter \ref{chapter:chapter_2}, we evaluated the accuracy of ten approaches using the two public available databases of healthy subjects IBSR18 and IBSR20. With the aim to include a wide set of different segmentation approaches and available tools, the analysis included well-known intensity based algorithms such as ANN, FCM, and KNN, and also public available toolboxes such as FAST, SPM5, SPM8, PVC, GAMIXTURE, SVPASEG, and FANTASM. Results were presented before and after correcting CSF masks, as it was shown that available annotations ignored sulcal CSF tissue on original masks. When sulcal CSF was corrected, SVPASEG, SPM8 and FAST yielded the highest accuracy on both databases.  However, most of the methods were variant to changes in acquisition sequences, intensity inhomogeneities or special attributes of the different databases,  which highlighted the fact that brain tissue segmentation problem was still open, because there was not a single method that achieved a significant higher accuracy on all tissues. 

Afterwards, six of these methods (ANN, FCM, FANTASM, FAST, SPM5 and SPM8) were evaluated on MS data from different hospital centers and scanners in order to analyze the extent to which tissue volume estimations were affected by changes in WM lesion volume and intensity. Our results showed that SPM8 was the method with the lowest differences in total volume, while FANTASM and again SPM8 were the methods where the incidence of WM lesions was the lowest on normal-appearing tissue. In general, differences in tissue volume were lower on methods combining morphological prior information such as SPM5 and SPM8, or spatial constraints such as FANTASM and FAST. In contrast, these differences were higher on simpler intensity based algorithms that lacked of spatial correction such as FCM and ANN. This fact and the higher performance on healthy data of former methods, stressed the necessity of adding morphological prior information and/or spatial constraints in automated brain tissue segmentation, not only to overcome inherent MRI artifacts but also as an important component to deal with WM lesions.

The main factor in the observed differences in tissue volume across methods was caused by lesion volume. Furthermore, WM lesion voxels tended to be classified as GM on images where the variation between lesion signal intensity and the mean signal intensity of normal-appearing WM was higher, which indicates a direct relationship between the differences in brain tissue volume and changes in lesion load and WM lesion intensity. However, lesion voxels had also a direct effect on the observed differences in GM and WM outside lesion regions. As already commented in Chapter \ref{chapter:chapter_3}, these differences are especially important because they highlight the bias introduced by WM lesions on the estimation of tissue volume that is not pathologically affected. Our analysis showed that if lesion voxels were not considered to compute brain volume, still methods tended to overestimate GM, specially on images with higher lesion load. Observed differences in normal-appearing tissue volume were important, because although lesion voxels could be reassigned to WM after segmentation, if these lesions were present in image segmentation, part of the bias was still present. Furthermore, differences in total tissue volume may be canceled between the errors produced in the same lesion regions and the effect of these voxels in normal-appearing tissue. This fact clearly shows the necessity to process WM lesion before segmentation.


\section{Effect of lesion filling in tissue segmentation}

In the last years, different techniques have been proposed to reduce the bias introduced by WM lesions on brain tissue volume measurements of MS images, mostly by in-painting WM lesions on T1-w with signal intensities similar to normal-appearing WM.  After reviewing the related available literature in Chapter \ref{chapter:chapter_4}, we classified existing methods by those that filled WM lesions using the \textit{local} intensities from the surrounding neighboring voxels of lesions, and those that used \textit{global} WM intensities from the whole brain to fill WM lesions. Although all these methods had been already validated separately, we performed a general comparison of all the available techniques in order to analyze their accuracy on the same 1.5T and 3T data and also to investigate its performance with different tissue segmentation techniques such as FAST and SPM8.

% fer un estat de l'art ens ha servit per proposar un nou metode. 
This analysis served as a basis to propose a new technique to refill WM lesions which was a compromise between \textit{global} and \textit{local} methods. In contrast to other existing techniques, the proposed method filled lesion voxels intensities with random values of a normal distribution generated from the mean WM signal intensity of each two-dimensional slice. Our results showed that when compared to other methods, our approach yielded the lowest deviation in GM and WM volume on 1.5T and 3T data when FAST tissue segmentation was used. When SPM8 tissue segmentation method was used, the performance of our lesion filling method was also very competitive, yielding the lowest differences or similar to the best method in GM and WM. In contrast to the rest of pipelines, differences in tissue volume between the same images filled with our method and afterwards segmented with either FAST or SPM8 were very low ($<0.1\%$), which indicates that the proposed strategy was equally efficient independently of the tissue segmentation chosen.

The proposed algorithm performed significantly better that local methods on images with higher lesion load. In contrast to \textit{global} methods, \textit{local} methods may be limited by the range of similar intensities coming from the neighboring voxels, which on images with a large lesions may be introducing a bias on GM and WM tissue distributions by the addition of a considerable number of voxels with similar intensities. Furthermore, the performance our approach was also better on images with high lesion load when compared with \textit{local} methods, specially on images with lower resolution such as 1.5T data, most probably because our method estimated the mean global NAWM intensity for each slice independently, being more sensible to reproduce possible changes in the intensities between slices.


\section{Effect of automating lesion segmentation and filling on  tissue segmentation}
\label{sec:label}

% aquí cal dir que masked no és suficient. 

As already said earlier, lesion filling has been shown to be an effective method to reduce the effects of these lesions on tissue segmentation. However, in all the lesion-filling approaches including ours, MS lesions have to be known a priori, which requires to delineate lesions manually. This was a clear limitation in terms of fully automatizing brain tissue on the presence of MS lesions, which motivated the evaluation of the effect of automated lesion segmentation on tissue segmentation. Although different automated tissue segmentation methods have been proposed, most of them are based on supervised learning, which require to explicitly train them usually with a large amount of labeled data. Labeled data may be not available, which had pointed out the interest of the community in unsupervised methods that can operate without prior data. As shown in Chapter \ref{chapter:chapter_5}, we compared two fully unsupervised pipelines that combined automated lesion segmentation and filling as a first step to understand the effect of fully processed images in tissue segmentation. 
 
Given the performance shown in our previous studies and its widely use in clinical studies, SPM8 was used as a reference tissue segmentation method to measure tissue volume on a set of 70 3T images of CIS patients. On these images, available manual expert annotations were employed to refill WM lesions before segmentation using the filling method of the pipeline, and were considered as ground-truth. Afterwards, we evaluated the differences in GM and WM volume between the set of filled images using manual annotations and the same images processed using different variations of the SLS and LST toolkits that differed in the level of manual intervention.  Evaluating different pipelines with distinct levels of automation  permitted us to analyze the effect of each of the automated process involved in the obtained differences in total and normal-appearing tissue volume.  

As already suggested in Chapter \ref{chapter:chapter_3}, this new analysis showed that the effect of lesions in total tissue volume was limited due to a canceling effect between the errors produced in the same lesion regions, and the effect of these voxels in normal-appearing tissue. In all the pipelines that incorporated automated lesion segmentation, most of the observed differences in normal-appearing tissue were produced by the effect of false positive lesion voxels that were already segmented without refilling them. In contrast, there was not a relevant correlation between the number of false positive lesion voxels and the observed differences in normal-appearing GM and WM, which suggested that most of these miss-classified voxels were actually WM before refilling them. The relationship between errors in automated lesion and tissue segmentation suggest also the importance of not only to keep reducing the number of missed lesions, but also stress the necessity of contextual spatial information of lesion regions in order to confine them in WM and hence reduce the effect of miss-classified voxels on tissue segmentation.

As shown in the results presented in Chapter \ref{chapter:chapter_4}, masking-out lesion voxels before tissue segmentation might not be optimal, as leaving lesion voxels out of the tissue distributions appears to increase the differences in tissue volume with respect to lesion filled images, even if these voxels are re-assigned to WM afterwards. However, although not optimal, masking lesion before segmentation has been found a valid alternative to reduce the effects of WM lesions in research and clinical settings, and only in the recent years, lesion filling techniques are been already applied on research and clinical studies. Regarding to this, our results show that at least with the evaluated data, the differences in tissue volume between images where expert lesion masks have been masked-out and the same images where lesions have been automatically segmented and filled, are similar on images with low lesion load $(<10ml)$. In contrast, from our experiments we observed that differences in tissue volume tend to increase with lesion load on masked-out images, while the increase of the error is more moderated on the fully-automated images. However, given the available data and maximum lesion load considered in our analysis ($<20ml$), these findings should be considered with care. 

In any case, our analysis points out the fact that automated lesion segmentation and filling methods reduced significantly the impact of WM lesions on tissue segmentation, and with a similar performance to the pipelines that required manual expert intervention. These results are relevant and validate that each of these automated processes can be useful not only in terms of time and economic costs, but also as active processes in fully automated tissue segmentation in the presence of WM lesions.

\section{Fully automated tissue segmentation of images containing WM lesions}
\label{sec:label}

Previous sections have stressed the necessity to deal with MS lesions before tissue segmentation, showing several general insights that can be useful for automated tissue segmentation of images containing lesions. The obtained results of the different evaluated methods in Chapter \ref{chapter:chapter_2} and \ref{chapter:chapter_3} have pointed out the superiority of methods that were benefited by morphological prior information or spatial constraints in automated brain tissue segmentation. More importantly, the results obtained in Chapter \ref{chapter:chapter_3} have evidenced the effect of WM lesion on tissue segmentation and the necessity to deal with MS lesions in order to reduce not only the bias produced by the same lesions but also the effect of these lesion voxels in normal-appearing tissue. In this scenario, we have proposed a new lesion filling technique that was very competitive with different databases and tissue segmentation methods, as shown in Chapter \ref{chapter:chapter_4}. Finally, we have shown in Chapter \ref{chapter:chapter_5} that the addition of unsupervised lesion segmentation and filling into already existing tissue segmentation pipelines reduced significantly the error in tissue volume when compared with previous pipelines where lesions were segmented as normal tissue. 

Following these insights, we have developed a novel multi-channel method designed to segment brain tissues in MRI images of MS patients. As explained in Chapter \ref{chapter:chapter_6}, this approach makes use of a combination of intensity, anatomical and morphological prior maps to guide the tissue segmentation. Tissue segmentation has been tackled based on a robust partial volume segmentation where WM outliers have been estimated and refilled before segmentation using a multi-channel post-processing algorithm also integrating  partial volume segmentation, spatial context, and prior anatomical and morphological atlases. Furthermore, the proposed method takes advantage of new affordable processors such as GPUs that reduce up to four times the execution time to register and segment tissue when compared to general purpose CPUs. This property makes this method useful for studies containing a large number of subjects to analyze.   

The proposed method has been quantitatively and qualitatively evaluated using different databases of images containing WM lesions. In order to analyze the extent to which T1-w and FLAIR modalities intervened in the obtained accuracy, the proposed method was run in all experiment using only T1-w or using both T1-w and FLAIR image sequences. As shown by the presented results, the proposed technique yielded competitive and consistent results in both general and MS specific databases without parameter tweaking. In the MRBrainS tissue segmentation challenge\footnote{http://mrbrains13.isi.uu.nl/}, our method combining both T1-w and FLAIR was the best non-supervised technique of the challenge, being ranked in the 7th position out of 31 participant methods. When only the T1-w modality was used, still the accuracy of the proposed method was clearly superior to general purpose methods such as FAST (ranked 21th) and SPM12 (best ranked 17th), even if those used both image modalities. In MS data, the performance of our method combining T1-w and FLAIR sequences was similar or better to the best evaluated pipeline incorporating lesion segmentation and filling. Obtained differences in tissue volume between images processed with the proposed algorithm and the same images where lesions were filled using expert lesion annotations, were lower that $0.15\%$ on all tissues, validating the overall capability of the proposed method to reduce the effects of WM lesions on tissue segmentation. 

In general, our results showed that the percentages of error in tissue volume of our proposed approach were low and similar in both databases. The percentages of error were the lowest when the FLAIR modality was used, which evidences that this image sequence has a direct effect on the efficiency of the algorithm, and consequently it should be used when available. However, the accuracy of the method using only the T1-w modality was also superior to other general purpose strategies, which also evidences that the improvement in tissue segmentation was not only generated by the addition of the FLAIR modality, but also by the combination of intensity, anatomical and morphological priors, and the use of an specific outlier algorithm with integrated lesion filling.


%%% Local Variables:
%%% mode: latex
%%% TeX-master: "../main"
%%% End:
